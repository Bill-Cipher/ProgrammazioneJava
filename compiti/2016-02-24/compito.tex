\documentclass{article}[10pt]
\usepackage[pdftex]{graphicx}
\usepackage{amsfonts}
\usepackage[italian]{babel}
\usepackage{graphicx}

%****************enlarge layout
\textheight     243.5mm
\topmargin      -20.0mm
\textwidth      480pt
\hoffset        -80pt
%*****************theorems and such
\newcounter{esnu}
\newenvironment{esercizio}{\medskip \noindent {\bf Esercizio\addtocounter{esnu}{1} \arabic{esnu}}}{}
\pagestyle{empty}
\newcommand{\liff}{\mathrel{\leftrightarrow}}   % Logical IFF Symbol
\newcommand{\metaiff}{\Longleftrightarrow}      %iff in metatheory

\begin{document}

%\begin{tabular}{llclcr}
% \hspace{-35pt} &{\bf COGNOME:} & \hspace{100pt}        &{\bf NOME:}    & \hspace{100pt}        &{\bf MATRICOLA:}%\hspace{35pt} \\
%\hline
%\end{tabular}
\begin{center} {\bf Esame di Programmazione II, 24 febbraio 2016}\end{center}
%\`

Si intende programmare un registro delle iscrizioni in palestra. Per esempio, il
seguente programma crea delle iscrizioni di tre utenti, le stampa in ordine cronologico
e poi le proietta sul 2015, ristampando quelle del 2015:

{\small\begin{verbatim}
public class Main {
  public static void main(String[] args)
      throws IscrizioneSovrappostaException, IscrizioneTroppoLungaException, IscrizioneVuotaException {

    Registro registro = new Registro();
    Utente u1 = new Utente("Fausto", "Spoto");
    Utente u2 = new Utente("Alan", "Turing");
    Utente u3 = new Utente("Noam", "Chomsky");
    registro.aggiungi(new Iscrizione(u1, Mese.SETTEMBRE, 2014, Mese.GENNAIO, 2015));
    registro.aggiungi(new Iscrizione(u1, Mese.SETTEMBRE, 2015, Mese.FEBBRAIO, 2016));
    registro.aggiungi(new Iscrizione(u1, Mese.MAGGIO, 2016, Mese.LUGLIO, 2016));
    registro.aggiungi(new Iscrizione(u3, Mese.AGOSTO, 2015, Mese.SETTEMBRE, 2015));
    registro.aggiungi(new Iscrizione(u2, Mese.DICEMBRE, 2015, Mese.NOVEMBRE, 2016));
    registro.aggiungi(new Iscrizione(u3, Mese.FEBBRAIO, 2016, Mese.AGOSTO, 2016));
    
    System.out.println(registro);

    Registro registro2015 = registro.proiettaSu(2015);

    System.out.println(registro2015);
  }
}
\end{verbatim}
}

\noindent
La sua esecuzione dovr\`a stampare:

{\small\begin{verbatim}
Fausto Spoto: dall'inizio di SETTEMBRE 2014 alla fine di GENNAIO 2015: 200 euro
Noam Chomsky: dall'inizio di AGOSTO 2015 alla fine di SETTEMBRE 2015: 100 euro
Fausto Spoto: dall'inizio di SETTEMBRE 2015 alla fine di FEBBRAIO 2016: 210 euro
Alan Turing: dall'inizio di DICEMBRE 2015 alla fine di NOVEMBRE 2016: 360 euro
Noam Chomsky: dall'inizio di FEBBRAIO 2016 alla fine di AGOSTO 2016: 245 euro
Fausto Spoto: dall'inizio di MAGGIO 2016 alla fine di LUGLIO 2016: 120 euro

Fausto Spoto: dall'inizio di SETTEMBRE 2014 alla fine di GENNAIO 2015: 200 euro
Noam Chomsky: dall'inizio di AGOSTO 2015 alla fine di SETTEMBRE 2015: 100 euro
Fausto Spoto: dall'inizio di SETTEMBRE 2015 alla fine di FEBBRAIO 2016: 210 euro
Alan Turing: dall'inizio di DICEMBRE 2015 alla fine di NOVEMBRE 2016: 360 euro
\end{verbatim}}

\begin{esercizio}
\textbf{[2 punti]}
Si scriva l'\texttt{enum} \texttt{Mese} con 12 costanti per i dodice mesi dell'anno,
da \texttt{GENNAIO} a \texttt{DICEMBRE}.
\end{esercizio}

\begin{esercizio}
\textbf{[5 punti]}
Si completi la seguente classe che descrive un utente di una palestra
(campi, costruttori e metodi aggiuntivi devono essere definiti \texttt{private}):

{\small\begin{verbatim}
public class Utente implements Comparable<Utente> {
  ...
  public Utente(String nome, String cognome) { ... }
  public boolean equals(Object other) { devono avere stesso nome e stesso cognome }
  public int hashCode() { non deve essere banale }
  public int compareTo(Utente other) { li mette in ordine per cognome e poi per nome }
  public String toString() { ritorna il nome, uno spazio e il cognome }
}
\end{verbatim}}

\end{esercizio}

\begin{esercizio}
\textbf{[14 punti]}
Si completi la seguente classe che descrive un'iscrizione in palestra
(campi, costruttori e metodi aggiuntivi devono essere definiti \texttt{private}).
Il costruttore richiede di specificare l'utente che si iscrive, il mese
a partire dal quale si iscrive (primo giorno del mese) e il mese fino a quando
si iscrive (ultimo giorno del mese). Per esempio, iscriversi da febbraio 2015 a marzo 2015
significa iscriversi dal primo febbraio 2015 al 31 marzo 2015. Si scrivano le classi
delle due eccezioni controllate lanciate nel costruttore.

{\small\begin{verbatim}
public class Iscrizione implements Comparable<Iscrizione> {
  ...
  public Iscrizione(Utente utente, Mese meseInizio, int annoInizio, Mese meseFine, int annoFine)
      throws IscrizioneTroppoLungaException, IscrizioneVuotaException {
    ...
    se il mese di inizio viene cronologicamente dopo il mese di fine, lancia una IscrizioneVuotaException
    se l'iscrizione va oltre i 12 mesi, lancia una IscrizioneTroppoLungaException
  }

  public Utente getUtente() { ritorna l'utente che si e' iscritto }
  public boolean equals(Object other) { devono avere stesso utente, stesso inizio e stessa fine }
  public int hashCode() { non deve essere banale }
  public int compareTo(Iscrizione other) {
    li mette in ordine per inizio. A parita' di inizio, li mette in ordine per utente.
    A parita' di inizio e di utente, li mette in ordine per fine
  }
  public String toString() {
    ritorna una stringa del tipo: "Fausto Spoto: dall'inizio di SETTEMBRE 2014 alla fine di GENNAIO 2015"
  }
  public boolean sovrappostaCon(Iscrizione other) { 
    determina se this e other sono sovrapposte nel tempo (anche parzialmente)
  }
  public int costo() {
    ritorna il costo dell'iscrizione: per 1 o 2 mesi costa 50 euro al mese,
    da 3 a 5 mesi costa 40 euro al mese, da 6 a 11 mesi costa 35 euro al mese, per 12 mesi costa 30 euro al mese
  }
  public boolean relativaAl(int anno) { determina se e' relativa all'anno indicato, anche parzialmente }
}
\end{verbatim}}

\end{esercizio}

\begin{esercizio}
  \textbf{[10 punti, facoltativo per chi ha gi\`a fatto il progetto degli anni passati]}
  Si completi la seguente classe, che rappresenta un registro di iscrizioni in palestra
  in ordine crescente,
  e si implementi l'eccezione controllata lanciata dal metodo \texttt{aggiungi}
  (campi, costruttori e metodi aggiuntivi devono essere definiti \texttt{private}):

  {\small\begin{verbatim}
import java.util.SortedSet;
import java.util.TreeSet;

public class Registro {
  private final SortedSet<Iscrizione> iscrizioni = new TreeSet<Iscrizione>();
  
  public void aggiungi(Iscrizione iscrizione) throws IscrizioneSovrappostaException {
    aggiunge l'iscrizione indicata al registro. Se ne e' gia' presente una
    sovrapposta e per lo stesso utente, lancia una IscrizioneSovrappostaException
  }

  public String toString() {
    ritorna la stringa delle iscrizioni di questo registro, in ordine, riportando il costo
    accanto a ciascuna iscrizione
  }

  public Registro proiettaSu(int anno) {
    ritorna un nuovo registro che contiene tutte e sole le iscrizioni di questo registro che
    riguardano l'anno indicato, anche parzialmente. Questo registro non viene modificato
  }
}
  \end{verbatim}}

\end{esercizio}

\end{document}
