\documentclass{article}[10pt]
\usepackage[pdftex]{graphicx}
\usepackage{amsfonts}
\usepackage[italian]{babel}
%****************enlarge layout
\textheight     243.5mm
\topmargin      -20.0mm
\textwidth      480pt
\hoffset        -80pt
%*****************theorems and such
\newcounter{esnu}
\newenvironment{esercizio}{\medskip \noindent {\bf Esercizio\addtocounter{esnu}{1} \arabic{esnu}}}{}
\pagestyle{empty}
\newcommand{\liff}{\mathrel{\leftrightarrow}}   % Logical IFF Symbol
\newcommand{\metaiff}{\Longleftrightarrow}      %iff in metatheory

\begin{document}

%\begin{tabular}{llclcr}
% \hspace{-35pt} &{\bf COGNOME:} & \hspace{100pt}        &{\bf NOME:}    & \hspace{100pt}        &{\bf MATRICOLA:}%\hspace{35pt} \\
%\hline
%\end{tabular}
\begin{center} {\bf Esame di Programmazione II, 24 febbraio 2012}\end{center}

\begin{esercizio}
\textbf{[2 punti]}
Un \texttt{Draw} \`e un rettangolo di caratteri. Per esempio, il seguente \`e un \texttt{Draw} di
altezza (\emph{height}) 4 e larghezza (\emph{width}) 6:
%
{\small
\begin{verbatim}
DrAw
123ABC
  $@@@
abcdef
\end{verbatim}
}

\noindent
Si noti che le righe di un \texttt{Draw} sono tutte lunghe tanto quanto la larghezza del \texttt{Draw}
e non hanno nessun new line alla fine. Per esempio, la prima riga del \texttt{Draw} visto sopra \`e
la stringa \texttt{DrAw\_\_} (con due spazi alla fine).
La terza riga \`e \texttt{\_\_\$@@@} (con due spazi all'inizio).

Si completi la definizione di un \texttt{Draw} data dalle seguente classe astratta:
%
{\small
\begin{verbatim}
public abstract class Draw {
  public abstract int getWidth();
  public abstract int getHeight();
  protected abstract String getRow(int num) throws IllegalArgumentException;

  @Override
  public final String toString() {
    // questo dovete scriverlo voi
  }
}
\end{verbatim}
}

\noindent
Lo scopo del metodo \texttt{getRow} (implementato nelle sottoclassi!) sar\`a di restituire la riga
numero \texttt{num} del \texttt{Draw} (dove \texttt{num} \`e 0 per la riga pi\`u in alto ed \`e
\texttt{getHeight() - 1} per quella pi\`u in basso). Tale metodo genera un'eccezione non controllata di classe
\texttt{java.lang.IllegalArgumentException} (classe gi\`a esistente in Java)
se \texttt{num} non sta dentro tali limiti. Tutto quello che dovete fare in questo
esercizio \`e di implementare
il metodo \texttt{toString()} che restituisce la concatenazione delle righe del \texttt{Draw}, andando
a capo alla fine di ciascuna riga.
\end{esercizio}

\begin{esercizio}
\textbf{[3 punti]}
Si definisca una sottoclasse astratta \texttt{Letter} di \texttt{Draw} i cui oggetti
hanno larghezza 7 e altezza 8. Queste dimensioni non devono essere
modificabili dalle sottoclassi di \texttt{Letter}.
\end{esercizio}

\begin{esercizio}
\textbf{[5 punti]}
Si definiscano delle sottoclassi finali di \texttt{Letter}, chiamate \texttt{H}, \texttt{E},
\texttt{L} e \texttt{O}, che implementano i seguenti \texttt{Draw} (si ricordi che \texttt{\_} \`e
solo un modo per evidenziare gli spazi):
%
{\small
\begin{verbatim}
_______      _______      _______      _______
_*___*_      _*****_      _*_____      _*****_
_*___*_      _*_____      _*_____      _*___*_
_*___*_      _*_____      _*_____      _*___*_
_*****_      _*****_      _*_____      _*___*_
_*___*_      _*_____      _*_____      _*___*_
_*___*_      _*_____      _*_____      _*___*_
_*___*_      _*****_      _*****_      _*****_
\end{verbatim}}

Si definisca quindi una sottoclasse finale \texttt{Star} di \texttt{Draw} che implementa il seguente
\texttt{Draw}:
%
{\small
\begin{verbatim}
X___X
_X_X_
__X__
_X_X_
X___X
\end{verbatim}}
%
\end{esercizio}
%
\begin{esercizio}
\textbf{[6 punti]}
Un \texttt{HorizontalDraw} \`e un \texttt{Draw} formato concatenando orizzontalmente uno o pi\`u
\texttt{Draw} (i suoi \emph{figli}). I figli posso avere altezze e larghezze diverse e
vengono tutti allineati in alto. Per esempio, l'\texttt{HorizontalDraw} con due figli, uno \texttt{Star}
e una \texttt{H}, \`e il seguente:
%
{\small
\begin{verbatim}
X___X_______
_X_X__*___*_
__X___*___*_
_X_X__*___*_
X___X_*****_
______*___*_
______*___*_
______*___*_
\end{verbatim}}

\noindent
Si noti che \texttt{Star} ed \texttt{H} hanno altezze diverse e le righe mancanti allo \texttt{Star} diventano spazi aggiuntivi.

Si completi la seguente implementazione di un \texttt{HorizontalDraw}:
%
{\small
\begin{verbatim}
public final class HorizontalDraw extends Draw {
  private final Draw[] children;

  public HorizontalDraw(Draw... children) throws IllegalArgumentException {
    if (children.length == 0)
      throw new IllegalArgumentException("HorizontalDraw: one Draw is needed at least");
    this.children = children;
  }
  // qui dovete continuare voi
}
\end{verbatim}}
%
\end{esercizio}
%
\begin{esercizio}
\textbf{[6 punti]}
Un \texttt{FrameDraw} \`e un \texttt{Draw} che inserisce un altro \texttt{Draw} (il \emph{figlio}) dentro
una cornice di \texttt{@}. Per esempio, il \texttt{FrameDraw} che ha come figlio l'\texttt{HorizontalDraw}
dell'esercizio precedente \`e:
%
{\small
\begin{verbatim}
@@@@@@@@@@@@@@
@X___X_______@
@_X_X__*___*_@
@__X___*___*_@
@_X_X__*___*_@
@X___X_*****_@
@______*___*_@
@______*___*_@
@______*___*_@
@@@@@@@@@@@@@@
\end{verbatim}}

Si completi la seguente implementazione di un \texttt{FrameDraw}:
%
{\small
\begin{verbatim}
public final class FrameDraw extends Draw {
  private final Draw child;

  public FrameDraw(Draw child) {
    this.child = child;
  }
  // qui dovete continuare voi
}
\end{verbatim}}
%
\end{esercizio}
%
\hrule
\hrule
\hrule
\mbox{}\\

Se tutto \`e corretto, il seguente \texttt{main}:
%
{\small
\begin{verbatim}
public class Main {
  public static void main(String[] args) {
    Draw h = new H(), e = new E(), l = new L(), o = new O(), star = new Star();
    Draw hello = new HorizontalDraw(star, h, e, l, l, o, star);
    Draw frame = new FrameDraw(hello);
    System.out.println(new HorizontalDraw(hello, frame));
  }
}
\end{verbatim}}

\noindent
stamper\`a:
%
{\small
\begin{verbatim}
X   X                                   X   X@@@@@@@@@@@@@@@@@@@@@@@@@@@@@@@@@@@@@@@@@@@@@@@
 X X  *   *  *****  *      *      *****  X X @X   X                                   X   X@
  X   *   *  *      *      *      *   *   X  @ X X  *   *  *****  *      *      *****  X X @
 X X  *   *  *      *      *      *   *  X X @  X   *   *  *      *      *      *   *   X  @
X   X *****  *****  *      *      *   * X   X@ X X  *   *  *      *      *      *   *  X X @
      *   *  *      *      *      *   *      @X   X *****  *****  *      *      *   * X   X@
      *   *  *      *      *      *   *      @      *   *  *      *      *      *   *      @
      *   *  *****  *****  *****  *****      @      *   *  *      *      *      *   *      @
                                             @      *   *  *****  *****  *****  *****      @
                                             @@@@@@@@@@@@@@@@@@@@@@@@@@@@@@@@@@@@@@@@@@@@@@@
\end{verbatim}}
%
\end{document}
