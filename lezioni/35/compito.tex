\documentclass{article}[10pt]
\usepackage[pdftex]{graphicx}
\usepackage{amsfonts}
\usepackage[italian]{babel}
%\`
%****************enlarge layout
\textheight     243.5mm
\topmargin      -20.0mm
\textwidth      480pt
\hoffset        -80pt
%*****************theorems and such
\newcounter{esnu}
\newenvironment{esercizio}{\medskip \noindent {\bf Esercizio\addtocounter{esnu}{1} \arabic{esnu}}}{}
\pagestyle{empty}
\newcommand{\liff}{\mathrel{\leftrightarrow}}   % Logical IFF Symbol
\newcommand{\metaiff}{\Longleftrightarrow}      %iff in metatheory

\begin{document}

%\begin{tabular}{llclcr}
% \hspace{-35pt} &{\bf COGNOME:} & \hspace{100pt}        &{\bf NOME:}    & \hspace{100pt}        &{\bf MATRICOLA:}%\hspace{35pt} \\
%\hline
%\end{tabular}
\begin{center} {\bf Semplificazione dell'esame di Programmazione II del 2 settembre 2013 (2 ore)}\end{center}

Il gioco del 15 consiste in una matrice quadrata di 16 tessere di cui una sola \`e
vuota e le altre 15 sono riempite con i numeri tra 1 e 15. Il gioco \`e risolto se
le tessere sono ordinate in senso crescente secondo una lettura per righe,
cio\`e se la matrice \`e nella configurazione:
%
\[\begin{array}{cccc}
 1 & 2 & 3 & 4\\
 5 & 6 & 7 & 8\\
 9 & 10 & 11 & 12\\
 13 & 14 & 15 & \\
\end{array}\]

Si intende adesso generalizzare questo gioco a una matrice di dimensioni generiche
$\mathit{width}\times\mathit{height}$. Le tessere non sono
pi\`u solo numeriche, ma possono pi\`u generalmente essere degli oggetti
con una relazione di ordinamento fra di loro. Ad esempio, un gioco
$4\times 3$ con tessere alfabetiche di lunghezza fra 1 e 5 \`e il seguente:
%
{\small
\begin{verbatim}
 swhv    kt     g  wohp 
sqvtc       nzwuo   evs 
  hkf    qb    lt    me 
\end{verbatim}
}

\noindent
Si noti che tale gioco non \`e risolto poich\'e le stringhe non sono in ordine
alfabetico secondo una lettura per righe, n\'e la casella vuota \`e in basso a destra.

Una tessera deve estendere la classe:
{\small
\begin{verbatim}
public abstract class Tessera implements Comparable<Tessera> {
  public abstract boolean equals(Object other);
  public abstract String toString();
}
\end{verbatim}
}

\noindent
e quindi avr\`a anche il metodo \texttt{abstract int compareTo(Tessera other)} ereditato
da \texttt{java.lang.Comparable<Tessera>}.

\begin{esercizio}
\textbf{[3 punti]}
Si completi la sottoclasse di \texttt{Tessera} che implementa una tessera numerica
etichettata con \texttt{num}:
%
{\small
\begin{verbatim}
public final class TesseraNumerica extends Tessera {
  private final int num;
  TesseraNumerica(int num) { this.num = num; }
  ...
}
\end{verbatim}
}

\noindent
Una tessera numerica \`e \texttt{equals} solo a una tessera numerica che contiene lo stesso numero.
L'ordinamento fra tessere numeriche avviene secondo l'ordine crescente del numero contenuto.
L'ordinamento fra una tessera numerica e un altro tipo di tessera
avviene secondo l'ordine alfabetico crescente del nome delle loro classi.
\end{esercizio}

\begin{esercizio}
\textbf{[3 punti]}
Si completi la sottoclasse di \texttt{Tessera} che implementa una tessera alfabetica
etichettata con \texttt{s}:
%
{\small
\begin{verbatim}
public final class TesseraAlfabetica extends Tessera {
  private final String s;
  TesseraAlfabetica(String s) { this.s = s; }
  ...
}
\end{verbatim}
}

\noindent
Una tessera alfabetica \`e \texttt{equals} solo a una tessera alfabetica
che contiene la stessa stringa (rispetto all'\texttt{equals} fra stringhe).
L'ordinamento fra tessere alfabetiche avviene secondo l'ordine alfabetico della
stringa contenuta. L'ordinamento fra una tessera alfabetica e un altro tipo di tessera
avviene secondo l'ordine alfabetico crescente del nome delle loro classi.
\end{esercizio}

\begin{esercizio}
\textbf{[2 punti]}
Una fattoria di tessere \`e un oggetto con un metodo che restituisce una tessera
a caso (non necessariamente diversa) ogni volta che viene chiamato:

{\small
\begin{verbatim}
public interface FattoriaDiTessere {
  public Tessera mkRandom();
}
\end{verbatim}
}

\noindent
Si completi una sua implementazione che genera tessere numeriche a caso, numerate
con un numero a caso fra $1$ e \texttt{max} inclusi:
%
{\small
\begin{verbatim}
public class FattoriaDiTessereNumeriche implements FattoriaDiTessere {
  public FattoriaDiTessereNumeriche(int max) { ... }
  ...
}
\end{verbatim}
}
\end{esercizio}

\begin{esercizio}
\textbf{[4 punti]}
Si completi un'implementazione di una fattoria che ritorna
tessere alfabetiche casuali, etichettate con
stringhe alfabetiche a caso di lunghezza a caso fra $1$ e $5$ inclusi:
%
{\small
\begin{verbatim}
public class FattoriaDiTessereAlfabetiche implements FattoriaDiTessere {
  public FattoriaDiTessereAlfabetiche() {}
  ...
}
\end{verbatim}
}
\end{esercizio}

\begin{esercizio}
\textbf{[10 punti]}
Un gioco \`e una matrice di tessere distinte (non \texttt{equals}), di cui una sola \`e
vuota. Un gioco
\`e risolto quando la tessera vuota \`e in basso a destra e le tessere non vuote
sono in ordine crescente secondo una lettura per righe.
Si completi l'implementazione di un gioco:
%
{\small
\begin{verbatim}
public class Gioco {
  ...
  public Gioco(FattoriaDiTessere fattoria, int width, int height) {
    // costruisce un gioco a caso della dimensione indicata, posizionando la casella vuota a caso
    // e creando le altre caselle a caso usando la fattoria indicata
    ...
  }

  @Override
  public String toString() { // restituisce una stringa come da esempi in basso
    ...
  }

  public boolean risolto() { // determina se questo gioco e' risolto
    ...
  }
}
\end{verbatim}
}

\end{esercizio}

Se tutto \`e corretto, il seguente programma:
%
{\small
\begin{verbatim}
public class Main {
  public static void main(String[] args) {
    Gioco gioco = new Gioco(new FattoriaDiTessereAlfabetiche(), 4, 4);
    System.out.println(gioco);

    FattoriaDiTessere f = new FattoriaDiTessereNumeriche(8);
    do {
      gioco = new Gioco(f, 3, 2);
      System.out.println(gioco);
    }
    while (!gioco.risolto());
  }
}
\end{verbatim}
}

\noindent
stamper\`a qualcosa del tipo:

{\small
\begin{verbatim}
 swhv    kt     g  wohp 
sqvtc       nzwuo   evs 
  hkf    qb    lt    me 
aotgf ehnlf dyaxc   dnc 

    3           8 
    6     1     2 

    5           1 
    2     7     8 

    .............
    [molti tentavivi non risolti]
    .............

    2     3     4 
    6     7       
\end{verbatim}
}

\noindent
terminando quindi con un gioco risolto.

\end{document}
